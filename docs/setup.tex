% !TEX root = ./main.tex

\fdusetup{
	style = {
		% font = times,
		% 西文字体(包括数学字体)
		%   font = garamond|libertinus|lm|palatino|times|times*|none
		%
		% cjk-font = fandol,
		% 中文字体
		%   cjk-font = adobe|fandol|founder|mac|sinotype|sourcehan|windows|none
		font-size = -4,
		hyperlink = none,
		% 超链接样式
		%   hyperlink = border|color|none
		bib-backend = biblatex,
		% 参考文献支持方式
		% 允许选项:
		%   bib-backend = bibtex|biblatex
		bib-resource = {main.bib},
		% 参考文献数据源
		% 可以是单个文件,也可以是用英文逗号 “,” 隔开的一组文件
		% 如果使用 biblatex,则必须明确给出 .bib 后缀名
		bib-style = numerical,
		% 参考文献样式
		% 允许选项:
		%   bib-style = author-year|numerical|<其他样式>
		% 说明:
		%   author-year  著者—出版年制
		%   numerical    顺序编码制
		%   <其他样式>   使用其他 .bst(bibtex)或 .bbx(biblatex)格式文件
		%
		% cite-style = {},
		% 引用样式
		% 默认为空,即与参考文献样式保持一致
		% 仅适用于 biblatex;如要填写,需保证相应的 .cbx 格式文件能被调用
		%
		% declaration-page = {declaration.pdf},
		% 插入扫描版的声明页 PDF 文档
		% 默认使用预定义的声明页,但不带签名
   },
	info = {
		title = {论文题目},
		title* = {English Title},
		author = {某\quad 某},
		author* = {Jane Doe},
		supervisor = {某\quad 某\quad 教~授},
		department = {数学科学学院},
		major = {计算数学},
		degree = academic,
		% 学位类型:
		%   degree = academic|professional
		student-id = {10010101010},
		% date = {2023 年 1 月 1 日},
		% 默认为编译日期
		instructors = {{某\quad 某\quad 教~授},{某\quad 某\quad 教~授},{某某某\quad 教~授}},
		keywords = {关键词1, 关键词2, 关键词3, 关键词4},
		keywords* = {KeyWords1, KeyWords2, KeyWords3, KeyWords4},
		clc = {O24},
	}
}

% 序号
\usepackage{enumerate}

% 定义图片文件目录与扩展名
\graphicspath{{figs/}}
\DeclareGraphicsExtensions{.pdf,.eps,.png,.jpg,.jpeg}

% 子图宏包
\usepackage{subcaption}
\usepackage{bicaption}

% 确定浮动对象的位置,可以使用 [H],强制将浮动对象放到这里(可能效果很差)
\usepackage{float}

% 表格排列
\usepackage{array}

% 固定宽度的表格
\usepackage{tabularx}

% 使用三线表:toprule,midrule,bottomrule。
\usepackage{booktabs}

% 表格中支持跨行
\usepackage{multirow}

% 使用长表格
\usepackage{longtable}

% 附带脚注的表格
\usepackage{threeparttable}

% 附带脚注的长表格
\usepackage{threeparttablex}

% 算法环境宏包
\usepackage[ruled,vlined,linesnumbered]{algorithm2e}
% \usepackage{algorithm, algorithmicx, algpseudocode}

% 代码环境宏包
\usepackage{listings}
\lstdefinestyle{lstStyleCode}{%
  aboveskip         = \medskipamount,
  belowskip         = \medskipamount,
  basicstyle        = \ttfamily\zihao{6},
  commentstyle      = \slshape\color{black!60},
  stringstyle       = \color{green!40!black!100},
  keywordstyle      = \bfseries\color{blue!50!black},
  extendedchars     = false,
  upquote           = true,
  tabsize           = 2,
  showstringspaces  = false,
  xleftmargin       = 1em,
  xrightmargin      = 1em,
  breaklines        = false,
  framexleftmargin  = 1em,
  framexrightmargin = 1em,
  columns           = flexible,
  keepspaces        = true,
  texcl             = true,
  mathescape        = true
}
\lstnewenvironment{codeblock}[1][]{%
  \lstset{style=lstStyleCode,#1}}{}

% 绘图宏包
\usepackage{tikz}
\usetikzlibrary{arrows.meta, shapes.geometric,cd}

% 超链接
\usepackage{xcolor}
\usepackage{hyperref}

% 页码显示
\usepackage{fancyhdr}
\pagestyle{fancy}
\fancyhf{} 
\fancyfoot[LE,RO]{\thepage} 
\fancypagestyle{plain}{
    \fancyhf{}
    \fancyfoot[LE,RO]{\thepage} 
}
% `L`, `C`, `R`分别对应左、中、右,`E`, `O`对应偶数页和奇数页


% 自定义引用样式,支持可选参数
\NewDocumentCommand{\mycite}{o m}{%
  \IfNoValueTF{#1}
    {\scalebox{1.2}[1.2]{\raisebox{-0.70ex}{\cite{#2}}}}
    { \scalebox{0.9}[0.9]{\raisebox{0.10ex}{\textnormal{[#1, {\scalebox{1.3}[1.3]{\raisebox{-0.80ex}{\citealp{#2}}}}]}}}}}

