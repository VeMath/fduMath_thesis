% !TEX root = ./main.tex

\fdusetup{
	style = {font-size = -4,
			bib-backend =bibtex,
            bib-resource = {main.bib},
			bib-style = numerical
   },
	info = {
		title = {论文题目},
		title* = {English Title},
		author = {某\quad 某},
		author* = {Jane Doe},
		supervisor = {某\quad 某\quad 教~授},
		department = {数学科学学院},
		major = {计算数学},
		student-id = {10010101010},
		instructors = {{某\quad 某\quad 教~授},{某\quad 某\quad 教~授},{某某某\quad 教~授}},
		keywords = {关键词1, 关键词2, 关键词3, 关键词4},
		keywords* = {KeyWords1, KeyWords2, KeyWords3, KeyWords4},
		clc = {O24}
	}
}

% 序号
\usepackage{enumerate}

% 定义图片文件目录与扩展名
\graphicspath{{figs/}}
\DeclareGraphicsExtensions{.pdf,.eps,.png,.jpg,.jpeg}

% 子图宏包
\usepackage{subcaption}
\usepackage{bicaption}

% 确定浮动对象的位置,可以使用 [H],强制将浮动对象放到这里(可能效果很差)
\usepackage{float}

% 表格排列
\usepackage{array}

% 固定宽度的表格
\usepackage{tabularx}

% 使用三线表:toprule,midrule,bottomrule。
\usepackage{booktabs}

% 表格中支持跨行
\usepackage{multirow}

% 使用长表格
\usepackage{longtable}

% 附带脚注的表格
\usepackage{threeparttable}

% 附带脚注的长表格
\usepackage{threeparttablex}

% 算法环境宏包
\usepackage[ruled,vlined,linesnumbered]{algorithm2e}
% \usepackage{algorithm, algorithmicx, algpseudocode}

% 代码环境宏包
\usepackage{listings}
\lstdefinestyle{lstStyleCode}{%
  aboveskip         = \medskipamount,
  belowskip         = \medskipamount,
  basicstyle        = \ttfamily\zihao{6},
  commentstyle      = \slshape\color{black!60},
  stringstyle       = \color{green!40!black!100},
  keywordstyle      = \bfseries\color{blue!50!black},
  extendedchars     = false,
  upquote           = true,
  tabsize           = 2,
  showstringspaces  = false,
  xleftmargin       = 1em,
  xrightmargin      = 1em,
  breaklines        = false,
  framexleftmargin  = 1em,
  framexrightmargin = 1em,
  columns           = flexible,
  keepspaces        = true,
  texcl             = true,
  mathescape        = true
}
\lstnewenvironment{codeblock}[1][]{%
  \lstset{style=lstStyleCode,#1}}{}

% 绘图宏包
\usepackage{tikz}
\usetikzlibrary{arrows.meta, shapes.geometric,cd}

% 超链接
\usepackage{hyperref}
